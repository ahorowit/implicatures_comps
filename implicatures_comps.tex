\documentclass[10pt,letterpaper]{article} 
\usepackage{cogsci} 
\usepackage{pslatex} 
\usepackage{apacite}
\usepackage{graphicx}
\usepackage{pdfsync}
\usepackage{natbib}


\title{Exploring sources of developmental change in children's pragmatic inferences about scalar terms}

\author{{\large \bf Alexandra C. Horowitz} \\ \texttt{ahorowit@stanford.edu}\\ Department of Psychology \\ Stanford University \\ 
\And {\large \bf Michael C. Frank} \\ \texttt{mcfrank@stanford.edu} \\ Department of Psychology \\ Stanford University \\ }

\begin{document}

\maketitle

\begin{abstract} 

The ability to compute implicatures--inferring that a weak statement (e.g. ``some of my siblings'') implies that a stronger statement (``all of my siblings'') could not be used--is a popular test case for children's pragmatic development. A growing body of work suggests that children are sensitive to speakers' intended meaning based on word choices, but their performance in experimental implicature tasks varies greatly depending on the context, methods, measures, and language used. We examine the development of children's ability to compute implicatures by combining different types of implicatures and control trials into a single, simple paradigm. We showed children three book covers with different combinations of pictures, and had them select which one they thought a speaker was describing. In Experiment 1, we included both ad-hoc (contextualized) and scalar (quantifier) descriptions, and found that 4-year-olds were at ceiling in ad-hoc trials but had difficulty with scalar implicatures (``some'') and nonexistence quantifier trials (``none''), and performance on these trials was highly correlated.  In Experiment 2, 4-year-olds' performance increased when we included only scalar trials, and we found a developmental increase across the preschool years and replicated a predictive relationship between ``some'' and ``none'' trials. In a simple, supportive paradigm, our work illustrates that preschoolers' recognition of implicatures relates both to their comprehension of particular descriptors used and also their recognition of relevant lexical alternatives. 



%replicated the finding that both ``some'' and ``none'' trials are difficult for yo ung children, and performance with these terms was bimodal and highly correlated. In a simple, supportive paradigm, our work illustrates that preschoolers' recognition of implicatures relates both to their comprehension of particular descriptors used, in addition their comparison with other possible lexical alternatives. 

{Keywords:} Pragmatics; implicature; language development. 
\end{abstract}

\section{Introduction}


Children's processing of scalar implicatures has been a focal case study for children's pragmatic development.  These generalized implicatures are formed from the use of a weaker member of a lexical scale used in place of a stronger alternative, such as quantifiers (\emph{some} versus \emph{all}), modals (\emph{possibly} versus \emph{definitely}), logical connectives (\emph{or} versus \emph{and}), and numerals (\emph{one} versus \emph{two}) \citep{horn1984}.  On the Gricean analysis \citep{grice1975}, the selection of a weaker scalar term implies that a stronger one could not be used.  For example, stating that ``I completed \emph{some} of my homework'' conveys that it would not be truthful or accurate to use the stronger description that I completed \emph{all} of my homework.  If I had completed it all, I should have said so.  Because I used the weaker quantifier, \emph{some}, this implies that I completed some \emph{but not all} of the work.  

Although adults spontaneously compute scalar implicatures along lexical scales, children's performance is variable even fairly late in development \citep{noveck2001}.  Subsequent studies suggest that Noveck's paradigm underestimates children's pragmatic reasoning \citep{guasti2005,papafragou2003, papafragou2004}.  More sensitive measures reveal that five-year-olds are able to perform similarly to adults when they were offered a ternary choice (given the choice between awarding a small, medium, or large prize, children were more likely to give a medium prize for weaker descriptions), suggesting that some study designs may mask the extent of children's reasoning \citep{katsos2011}.  

The collective contributions from research on children's abilities to compute scalar implicatures suggests that their fragile performance may have less to do with their general pragmatic knowledge per se, and more to do with the constraints of experimental paradigms. The Alternatives Hypothesis, proposed by Barner and colleagues \citep{barner2010, barner2011}, posits that children's ability to compute scalar implicatures relies on their recognition of the relevant scalar alternatives (e.g. that use of the weaker term ``some'' conveys a direct contrast with the stronger alternative ``all'', thus implying \emph{some \textbf{but not all}}).  In other words, children's pragmatic inferences rely on their ability to consider relevant possible alternative word choices that could have been used in place of the ones the speaker chose.  

Support for this hypothesis comes from evidence that children's performance in implicature tasks increases when they have stronger access to lexical alternatives.  Enriched contexts such as evaluating performance in competitions \citep{papafragou2003} and restrictive language such as ``only'' helps constrain preschoolers' logical tolerance for upper-bounded interpretations \citep{barner2011} of implicatures. Additionally, even three-year-olds show evidence of computing implicatures for particularized scales that depend on visual features in a context \citep{stiller2014}, and they can even identify that weak quantifiers imply a subset when prosodic emphasis is sued (preschoolers selected a subset of faces when asked to ``make SOME faces happy.'' but not ``make some faces HAPPY'') \citep{miller2005}. These findings suggest that, in line with the Alternatives Hypothesis, supportive contextual framing, linguistic framing, contrastive stress, and familiarity with other possible lexical alternatives help children recognize implied meaning conveyed through production choices when they recognize what alternatives \emph{could} have been used, but were not. 

Although the case study of scalar implicatures has been widely studied as an example of children's pragmatic reasoning appearing very different from that of adults, the body of work to date has varied widely by the scales used, ages tested, linguistic framing, and paradigm design.  These confounding factors make it difficult to reconcile all of the disparate findings, but the Alternatives Hypothesis provides an elegant explanations for children's reasoning across the multitude of scalar implicature tasks.  

In order to investigate the development of preschoolers' ability to compute implicatures, we designed a simple reference selection task in which children were asked to select with of three book covers they thought the experimenter was describing. Our design allowed us to a) fully counterbalance the instructions children heard across trials, including ad-hoc versus scalar descriptions as well as implicature versus unambiguous control targets, b) examine both within-subject patterns of responses as well as between-subject developmental patterns, c) provide children with visual alternatives to help reduce the demands of the task, and d) provide lexical alternatives across the duration of the 18 trials of the task. 

In Experiment 1, we included both ad-hoc and scalar descriptions with implicature and control trials for each. We replicated the findings that preschoolers were at ceiling across ad-hoc trill types, and also that their performance was significantly reduced for scalar implicature trials. We were surprised to find a bimodal and predictive relationship between individuals' performance on ``some'' (scalar implicature) and ``none'' (scalar control) trials, suggesting that the ability to compute scalar implicatures may involve not only recognition of the stronger scalemate ``all'', but with both ends of the scale. In Experiment 2, we ran the same task but replaced the ad-hoc trials so that all of the descriptions included scalar quantifiers. In this version of the task, we still found a highly correlated relationship between performance on ``some'' and ``none'' trials, an overall increase in 4-year-olds' performance, and a developmental increase across trial types in the task with age. Overall, our findings suggest that scalar implicatures are difficult for preschoolers even in supportive contexts, and stronger recognition of lexical alternatives boosts performance. 

%In our current work, we combine a range of implicatures within a single paradigm design to examine whether we find reliable within- and between-subjects developmental patterns predicted by the availability of lexical alternatives. 


%This is the first study is to use a single paradigm to test a battery of implicatures, allowing us to examine both within-subject differences in performance by implicature type, as well as compare between-subjects data to investigate whether similar developmental patterns emerge across different types of implicatures with age.  In Experiment 1, we show that 4-year-olds have no difficulty with 


	

\section{Experiment 1} 

\subsection{Methods}
\subsubsection{Participants}

A planned sample of 48 children were recruited from Bing Nursery School at Stanford University.  Participants were grouped into two age groups: 24 4.0- to 4.5-year-olds (M = 4;2) and 24 4.5- to 5.0-year-olds (M = 4;7). Two additional children were excluded for stopping the study early, and one due to an experimenter error. 
\subsubsection{Stimuli}

Children were shown printed images of three clip art book covers, each depicting four familiar items. An initial training trial featured a single unique item on each cover. For each of the 18 test trials, one book contained four items of a kind (e.g. four dogs), one book contained a different set of four items (e.g. four cats), and one book contained two pictures of a previous set and two pictures of a different set (e.g. two cats and two birds) (see Figure \ref{fig:demo}). Each set of books featured a different set of familiar items. The trials were presented in a fixed order, counterbalanced for target location and book triad positions.  The description condition (ad-hoc or scalar quantifier) and trial type (implicature or control) for each book set were randomized across participants. 

\begin{figure}[t] 
  \begin{center} 
    \includegraphics[width=3.5in]{figures/implicatures_demo_letters.png} 
    \caption{\label{fig:demo} Example trial image set. Children saw three book covers with familiar images. The experimenter provided a clue about which of the three covers she was thinking of, using either an ad-hoc or scalar description of either an unambiguous or implicature target (see scripts in Table \ref{tab:scripts}).  
    }
    \end{center} 
\vspace{-1ex} 
\end{figure}

 \begin{table*} [t]
   \caption{An outline of the study designs for Experiments 1 and 2, using script examples for the trial set pictured in Figure \ref{fig:demo}.  \label{tab:scripts} } 
   \begin{center} 
     \begin{tabular}{lccccc} 
        %  & \multicolumn{4}{c}{} \\
                      \hline 
       \null   Condition  & Trial type & \# trials in Expt. 1 & \# trials in Expt. 2 & Statement: ``On the cover of my book, ...'' & Target   \\ 
       \hline  
            Scalar & implicature & 4 & 6 &  ``...some of the pictures are cats'' & B	 \\ 
          & all  & 2 &  6 & ``...all of the pictures are cats'' & C		                 \\
           & none  & 2 & 6 & ``...none of the pictures are cats'' & A			\\ 
               & unambiguous `some' 	&  2 & 0 & ``...some of the pictures are birds'' & B					        \\ 
	\hline
	    Adhoc       & implicature & 4 & 0 & ``...there are cats'' & C 		\\ 
	     & distractor & 2 & 0 & ``...there are dogs'' & A	     \\ 
          & comparison & 2 &  0 & ``...there are birds'' & B 	   \\
       \hline 

     \end{tabular} 
  \end{center}
 \end{table*}
 

\subsubsection{Procedures}

Participants were tested individually in a quiet room at their preschool.  The experimenter explained that they would be playing a game in which she would think about one of the three books on each page and give a clue about it. She emphasized that she would only give one clue for each page, so children were to make their best guess about which book she was describing based on that clue. A breakdown of the trial types and sample scripts is provided in Table \ref{tab:scripts}.

Children began the task with a training trial where each of the covers featured a single unique item (e.g. ``On the cover of my book, there's a TV'' and children chose between a book with a TV, a book with a sofa, and a book with a refrigerator).  Following this initial training, children saw 18 test trials with new sets of familiar items similar to the one depicted in Figure \ref{fig:demo}.  Children were instructed to point to the book they thought the experimenter was describing. If children pointed to more than one book or their response was unclear, they were reminded that the experimenter was talking about just one book, and asked to touch the one book they thought she meant.  

For ad-hoc trials, the experimenter described the book cover by naming the images pictured. Ad-hoc control trials (four total) referred to unambiguous targets (e.g. ``On the cover of my book, there are dogs/birds'', see Figure \ref{fig:demo}).  Ad-hoc implicature trials (four total) required an inference about the speaker's intended meaning: e.g. ``On the cover of my book, there are cats'' could potentially refer to either the book with only cats or the book with cats and birds, but the speaker's decision to describe only cats suggests that she is referring to the cover with all cats and no birds, or else she would have mentioned both types of animals. 

For scalar trials, the experimenter described the target using quantifiers. Scalar control trials (6 total) referred to unambiguous targets: two trials each using \emph{all} and \emph{none} (e.g.``On the cover of my book, all/none of the pictures are cats'') and two trials featuring an unambiguous referent of \emph{some} (e.g. ``On the cover of my book, some of the pictures are birds'').  On scalar implicature trials (four total), the experimenter used a weak quantifier in reference to the item pictured across two book covers (e.g. ``On the cover of my book, some of the pictures are cats''). Because the speaker used a weak quantifier, the implicature is that she must mean the cover with two cats and two birds, because if she had meant the cover with only cats, she would have use the stronger quantifier (\emph{all}) instead. 

The script orders were counterbalanced across four lists, and the conditions (ad-hoc or scalar) and trial types (implicature or control) were spaced as much as possible so that two trials of the same type never occurred twice in a row. Children enjoyed the task, responded quickly to the clues, and often made statements such as, ``I'm good at this!'' although they were not provided feedback about their selections. The test session took about ten minutes to complete.

\subsection{Results and Discussion}


Results are presented in Figure \ref{fig:expt1}. Responses were coded as correct if children selected the intended target of the description. Across all of the ad-hoc condition trial types (pink bars), both younger and older 4-year-olds were near ceiling in selecting the intended target. This finding replicates previous work indicating that preschoolers can compute ad-hoc implicatures \citep{stiller2014} in a new task, and suggests that children are sensitive to a speaker's contextual descriptive choices even in the presence of varied types of descriptions (control trials and scalar references). 

Children's performance on scalar trials was markedly different. Although they were at ceiling on control \emph{all} trials, we found that children were overall at chance for \emph{some} and \emph{none} trials.  Examining their patterns of performance more closely, we found significant bimodal distributions for both \emph{some} (D=0.15, p$<$0.0001) and \emph{none} (D=0.20, p$<$0.0001) trials, indicating that individuals tended not to respond at chance, but either consistently correctly or incorrectly on these categories of trials. Additionally, children's success on \emph{some} and \emph{none} trials were highly correlated with each other (r=-0.47, p$<$0.001), such that children who performed better on \emph{some} trials also tended to perform better on \emph{none} trials (see Figure \ref{fig:expt1scatterplot}).  These findings suggest that children's difficulty on scalar implicature trial may not be wholly based not their consideration of the stronger alternative \emph{all}, but might be due to familiarity with the both ends of the quantifier scale (\emph{none--some--all}). They may need recognition of both extremes of the scale before consistently identifying the meaning of the intermediate, \emph{some} term. 

%We were surprised by children's responses on scalar trials. 
To examine the reliability these patterns of responses, we ran a logistic mixed effect model, predicting correct responses as the interaction of age, condition (ad-hoc or scalar) and trial type (implicature or control), with random effects of participant and trial type. Performance was marginally lower for scalar trials than ad-hoc trials ($\beta = -8.02$, $p =0.09$), and there was a significant interaction between condition and trial type, such that performance was significantly worse on scalar implicature trials ($\beta = 16.45$, $p = 0.02$).  There was also a significant 3-way interaction between condition, trial type, and age, such that performance on scalar implicature trials decreased with age ($\beta = -4.16$, $p < 0.01$). 

Overall, we found that scalar implicatures were hard for children in our task, and we wondered why this might be the case because we had tried to reduce as many task demands as possible in our task. Despite the presence of both visual alternatives (via the three selection choices) and lexical alternatives (conveyed across trials), children were at chance in their selections on scalar implicature trials. We were also surprised to find a developmental change in children's responses on \emph{none} trials. 
%Most previous work examines children's inferences about weak quantifiers (\emph{some}) compared with stronger alternatives (\emph{all}) without included the other end of the scale (\emph{none}).  
We had expected \emph{none} trials to serve as a simple unambiguous control, but quickly realized that this term may be difficult for young children and perhaps may contribute to children's difficulty computing scalar implicatures from weak quantifiers. 

Although a goal of our study was to examine pragmatic development by comparing children's performance across a variety of inference trials, we wondered if including \emph{both} ad-hoc and scalar quantifier descriptions led children to form expectations about the speaker that influenced their response. For instance, we were concerned whether their overwhelming success on ad-hoc implicature trials (e.g. ``On the cover of my book, there are cats'') might lead children to misinterpret the intention of ``some'' in scalar implicature trials from ``On the cover of my book, some of the pictures are cats'' to ``On the cover of my book, there are some cats.'' If children are forming expectations about the speaker that override their sensitivity to the particular word choices in the referential expression, then their performance may be biased by the presence of the ad-hoc trials.  To investigate this idea, we removed ad-hoc trials and ran a scalar-only version of the study in Experiment 2. 

\begin{figure*}[t] 
  \begin{center} 
    \includegraphics[width=7.5in]{figures/implicatures_adhocScalar_long.pdf} 
    \caption{\label{fig:expt1} Proportion of correct responses per age group for control and implicature trials for ad-hoc (pink) and scalar (blue) conditions in Experiment 1.    }
    \end{center} 
\vspace{-1ex} 
\end{figure*}

\begin{figure}[h] 
  \begin{center} 
    \includegraphics[width=3.5in]{figures/implicatures_adhocScalar_scatterplot.pdf} 
    \caption{\label{fig:expt1scatterplot} Scatterplot relating individuals' performance on \emph{some} trials and \emph{none} trials per age group in Experiment 1. The main effect is plotted in black, and correlations per age group illustrated by the dotted lines.    }
    \end{center} 
\vspace{-1ex} 
\end{figure}


	


\section{Experiment 2} 

%In Experiment 1, we examined preschooler's sensitivity to speakers' descriptive choices by combining ad-hoc and scalar quantifier descriptions into a single, supportive task. We found that children had no trouble with ad-hoc control or implicature trials, but they did have difficulty with scalar implicature trials and descriptions involving \emph{none}. Although we tried to reduce as many task demands as possible by having children select the speaker's referent from visual alternatives, these simplified paradigm was still a challenge for children's ability to compute scalar implicatures. 

In Experiment 2, we ran a version of the task in which all 18 trials featured scalar quantifiers in order to investigate whether preschoolers would show increased sensitivity to individual quantifier use when it was the only descriptive variable that changes across trials. We extended our age range to 3- to 5-year-olds broken into half-year age groups to examine the developmental patterns of scalar quantifier comprehension more precisely. 

\subsection{Methods}

\subsubsection{Participants}

We recruited a new sample of participants from Bing Nursery School, broken into half-year age groups: 10 3.0--3.5 year-olds (M=3;4), 12 3.5--4.0 year-olds (M=3;8), 14 4.0--4.5 year-olds (M=4;3), and 12 4.5--5.0 year-olds (M=4;8).  One additional child was excluded for stopping the task early. 

\subsubsection{Stimuli}

The same materials were used as in Experiment 1.  The only changes made were to the scripts, such that ad-hoc trials were removed and all trials were converted into scalar quantifier references.  The 18 test trials contained six control \emph{all} trials (e.g. ``On the cover of my book, all of the pictures are cats''), six control \emph{none} trials (``On the cover of my book, none of the pictures are cats''), and six scalar implicature \emph{some} trials (``On the cover of my book, some of the pictures are cats'').  Trials were presented in a fixed order, counterbalanced for target location and triad order.  There were three scripts randomized across participants, with each trial type (\emph{all, none} or \emph{some}) occurring once for each book set across the lists, and with a pseudo-randomized trial order such that participants never heard the same trial type twice in a row. See Table \ref{tab:scripts} for an outline of the design and scripts.

\subsubsection{Procedures}

Other than the change to the scripts, the procedure was identical to Experiment 1. 

\subsection{Results and Discussion}

The proportion of correct trials by age group per trial type is presented in Figure \ref{fig:expt2}.  Children's performance increased with age for all trial types. All age groups were strongest on the \emph{all} trials, and the oldest children (4.5--5.0 year-olds) were the only age group above chance for all trial types.  

Although 4-year-olds' performance increased on \emph{some} and \emph{none} trials with the scalar-only descriptions of Experiment 2, children overall had more difficulty with these trials than for \emph{all} trials. We also still found significant binomial patterns of responses for both \emph{some} (D=0.12, p$<$0.0001) and \emph{none} (D=0.15, p$<$0.0001) trials. Individuals' performance on \emph{none} and \emph{some} trials were highly correlated (r=0.52, p$<$0.001). The correlation between \emph{none} and \emph{some} performance in Experiment 2 is depicted in \ref{fig:expt2scatterplot}.

We ran a logistic mixed effect model, predicting correct responses as the interaction of age and trial type (\emph{none, some} or \emph{all}), with random effects of participant and trial type. The only significant effect that emerged was age, such that performance increased across trials as children got older ($\beta = 3.80$, $p < 0.001$). 

Our pattern of results suggest that children's ability to interpret scalar quantifiers is largely driven by their familiarity with these terms, which is highly related to age.  Overall, children's success in selecting the speaker's intended target increased as children got older, perhaps driven by their knowledge of quantifier terms, or perhaps due to their sensitivity to pragmatic implications of word choice more generally. The bimodal and highly predictive pattern of responses for both \emph{none} and \emph{some} terms, and simultaneous near ceiling performance on \emph{all} trials, suggests that children may be developing their specific understanding of quantifier use in their preschool years. They may be learning that both \emph{none} and \emph{some} contrast with \emph{all} in parallel, as evidenced by their systematic responses to these trial types. Each of the three trial types (\emph{all, some}, and \emph{none}) occurred with equal frequency in Experiment 2, so a highly correlated, bimodal pattern of responses between \emph{none} and \emph{some} trials indicates predictive relationship. 


\begin{figure}[t] 
  \begin{center} 
    \includegraphics[width=3.5in]{figures/implicatures_scalarOnly_clean.pdf} 
    \caption{\label{fig:expt2} Proportion of correct responses per age group for each of the scalar trial types (\emph{all, some} and \emph{none}) in Experiment 2.     }
    \end{center} 
\vspace{-1ex} 
\end{figure}

\begin{figure}[h] 
  \begin{center} 
    \includegraphics[width=3.5in]{figures/implicatures_scalarOnly_scatterplot.pdf} 
    \caption{\label{fig:expt2scatterplot} Scatterplot relating individuals' performance on \emph{some} trials and \emph{none} trials per age group in Experiment 2. The main effect is plotted in black, and correlations per age group illustrated by the dotted lines.  }
    \end{center} 
\vspace{-1ex} 
\end{figure}


 \section{General Discussion} 
 
We designed a simple task that allowed us to test children's sensitivity to a variety of word choice cues in a single paradigm, allowing us to investigate patterns of development both within- and between-subjects. We tried to minimize task demands as much as possible by using a reference selection context among three visual alternatives. In Experiment 1, we replicated the finding that preschoolers are strong at computing ad-hoc implicatures, and they were also at ceiling on unambiguous ad-hoc control trials. In Experiment 2, we found that preschoolers' recognition of all scalar terms in the task increased with age, and that removing the ad-hoc trials increased older children's performance on \emph{some} and \emph{none} trials. Our findings suggest that 4-year-olds are able to compute scalar implicatures, but their sensitivity to these inferences may be fragile and rely on strong contextual cues. 

Our work contributes to the existing literature in a number of ways: first, it offers a novel, simple paradigm that is less complicated than many other implicature tasks, leading us to feel more confident that our results reflect children's true sensitivity rather than inadvertent difficulties of challenging task demands.  Each test set remained visible to children, and they were merely asked to select which picture they thought was the referent of the speakers' description, which was randomized across trials. Second, the relatively high number of trials both helped strengthen our analytical power, and also helped offer children an opportunity to recognize lexical alternatives the the study progressed.  Although we did not find differences in children's performance from the start to the end of the task, we wanted to investigate this factor as a potential influence in children's performance. Third, we were able to not only compare performance across age groups, but our inclusion of both ad-hoc and scalar trials with implicature and control trials for each allowed us to also examine individual patterns of responses. This design helped us determine that preschoolers' performance on scalar implicature trials was bimodal, and highly related to their performance on \emph{none} trials, which we would have been unlikely to uncover in a purely between-subjects implicature design without controls. 

Our evidence largely supports the Alternatives Hypothesis, though suggests that there may be more to the picture of children's ability to compute scalar implicatures. Our ad-hoc trials in Experiment indicate that preschoolers had no difficulty making inferences about contextual descriptions, giving support that they evaluated the speaker's intended meaning when able to consider the possible descriptive choices for the visual alternatives; children tailored their selections to the speaker's most likely referent based on her choices of words given the potential items to be named in the scene. 

For the scalar trials, we also see evidence that performance was based on recognition of lexical alternatives: preschoolers' ability to compute scalar implicatures increases with age (presumably a proxy for familiarity with these scalemates). Additionally, the correlated bimodal pattern of responses for \emph{some} and \emph{none} trials in both Experiments 1 and 2 indicates that children's inferences are related to their comprehension of related lexical alternatives. Overall, these patterns of results support the idea that children's ability to compute implicatures relates to their ability to consider what other possible utterance choices a speaker could have used instead.  As children gain more experience with language and members of the same quantifier Horn scale, they are more likely to succeed in making inferences about the implications of these particular word choices. 

A pattern more difficult to reconcile with the Alternative Hypothesis is the finding that children's performance did not change over the course of the experiments. We had expected that, if children are sensitive to opportunities for inferences based on contrasts in lexical choices, that they should alter their behavior throughout the experiment. In other words, even if children were not complete comprehends of ``some'' and ``none'', they should realize that, across the first few trials, that if they know the meaning of \emph{all} (at which they perform near ceiling), then they should be able to make the inference that a change in quantifier (``some'' or ``none'' in place of ``all'') should indicate a change in meaning (that another quantifier should not \emph{also} refer to the picture of \emph{all}).  This concept could be inferred as purely a lexical inference (that ``all'' and ``some'' or ``none'' should not overlap in meaning), or also a pragmatic inference (that if the speaker has selectively used ``all'' on certain trials and uses ``some'' or ``none'' on other trials, she is probably intended a different meaning by using these different terms).  Either of these routes to inference should lead children to be more likely to select the picture that is \emph{not} ``all'' on ``some'' and ``none'' trials, which would likely lead to an increase in correct target selections for ``some'' trials because the target is the other book that contains the named items (rather than ``none'' trials, which require selecting the negated images named). Because we don't see a change in performance across trials, it may be the case that children need stronger lexical knowledge of particular terms before making inferences purely on contrastive word choice. It may not be enough just to hear that a speaker is using different words, and may be the case that children need more evidence about the particular meanings of these terms before they can demonstrate inferences about changes in their implied meaning. 

Sensitivity to speaker's word choices can facilitate the information exchange between conversational patterers. Implicatures are particularly useful instances of pragmatic inferences because they involve \emph{implied} meaning, offering opportunities for a inference with high likelihoods of success when recognized compared with more subtle uses of language. Our work suggests that children are sensitive to the implications of some lexical choices speakers make (such as in the case of ad-hoc implicatures), but their ability to pick up on all word choice cues takes longer to develop.  Children's computation of scalar implicatures increases in supportive contexts, but their inferences are fragile and seem to depend more on their knowledge of the lexical items than the pragmatic cues of varying descriptive choices. Children may need to have a stronger comprehension of a lexical category before making inferences about possible within-category distinctions word choices convey.  Once they can recognize the relevant category contrasts intended, they can compute implicatures from these word choices. 

\section{Acknowledgments}

Special thanks to the staff and families at the Bing Nursery School, and to Sara Altman and Carson Kautz for their help with data collection. 

\bibliographystyle{apacite}

\setlength{\bibleftmargin}{.125in} \setlength{\bibindent}{-\bibleftmargin}

\bibliography{implicatures}

\end{document}

